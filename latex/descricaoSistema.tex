\documentclass[oneside]{article}

\usepackage[utf8]{inputenc}
\usepackage{color} %Pacote utilizado para que seja possível adicionar cores ao texto
\usepackage{hyperref} %Pacote utilizado para adicionar bookmarks no PDF
\usepackage{fancyhdr}

\begin{document}
\title{Projeto para Disciplina de SIG \\
        \large Sistema Gegráfico para Suporte no Comércio de Venda Direta}
\addcontentsline{toc}{chapter}{Capa - Roteiro, Anotações}
\author{Eduardo Dorneles Ferreira de Souza}
%\date{2 de Julho de 2016}
\maketitle

\pagestyle{plain}

\section{Introdução}
Um modelo de negócio que está consolidado há alguns anos no mercado mundial é o
comércio produtos através da venda direta. A empresa pioneira nesse modelo de
negócio foi a editora responsável pela enciclopédia Britânica, a venda de porta
em porta foi um sucesso para a empresa e para diversas outras que também adotaram
o mesmo modelo. Atualmente, a venda direta continua sendo um modelo de negócio
fortemente explorado por diversas empresas, segundo o levantamento feito pela ABEVD
(Associação Brasileira de Vendas Diretas) em 2015 essa abordagem registrou
R\$ 19,5 bilhões de reais em volume de negócio e conta com mais de quatro milhões
de pessoas.

O uso de informações geográficas pode ser de grande utilidade para vendedores
e também para consumidores de produtos de marcas vendas diretas, por exemplo,
nem sempre um determinado vendedor possui um produto de interesse do cliente,
portanto seria interessante ele saber qual vendedor possui um produto de seu
desejo e onde ele se localiza.

Nesse projeto, propõe-se o desenvolvimento de uma ferramenta para suporte ao
mercado de vendas diretas que faça uso de informações geográficas a



\end{document}
