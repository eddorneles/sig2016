\documentclass[oneside]{article}

\usepackage[utf8]{inputenc}
\usepackage{color} %Pacote utilizado para que seja possível adicionar cores ao texto
\usepackage{hyperref} %Pacote utilizado para adicionar bookmarks no PDF
\usepackage{fancyhdr}

\begin{document}
\title{Projeto para Disciplina de SIG \\
        \large Sistema Geográfico para Suporte no Comércio de Vendas Diretas}
\addcontentsline{toc}{chapter}{Título --- Projeto para Disciplina de SIG}
\author{Eduardo Dorneles Ferreira de Souza}
%\date{2 de Julho de 2016}
\maketitle

\pagestyle{plain}

\section{Introdução}

O modelo de negócio de vendas diretas está consolidado há mais de um século no
mercado mundial. A empresa pioneira nesse modelo de foi a editora da enciclopédia
Britânica no final do século XVIII. A venda de porta em porta foi um sucesso
para a empresa e com o passar dos anos diversas outras que também adotaram
o mesmo modelo. Atualmente, a venda direta continua sendo um modelo de negócio
muito promissor para várias empresas, segundo o levantamento feito pela ABEVD
(Associação Brasileira de Vendas Diretas) em 2015 essa abordagem registrou
R\$ 19,5 bilhões de reais em volume de negócio e conta com mais de quatro milhões
de pessoas trabalhando ativamente.

O uso de informações geográficas pode ser de grande utilidade para vendedores
se ele tiver informações de quais localidades possuem maior potencial de venda
e também para consumidores de marcas de vendas diretas, por exemplo: o cliente
pode consultar um produto desejado e saber qual o representante mais próximo
que possua produto.

Nesse projeto, propõe-se o desenvolvimento de uma ferramenta para suporte ao
mercado de vendas diretas que faça uso de informações geográficas.

\section{Descrição Geral}
O sistema será capaz de gerir produtos de uma empresa (marca). Cada empresa
possui um conjunto de representantes (por exemplo: vendedores).
Dados relevantes sobre o representante:
\begin{itemize}
    \item Nome;
    \item RG;
    \item CPF;
    \item Endereço;
    \item Localização atual (caso disponibilize via dispositivo móvel);
    \item Número telefônico;
    \item Produtos disponíveis.
\end{itemize}
Dados relevantes sobre os produtos:
\begin{itemize}
    \item Nome;
    \item Preço;
    \item Descrição;
    \item Linha (ex: Profissional ou Home Care);
    \item Categoria (ex: Face, Corporal, etc);
    \item Subcategoria (classificação mais específica do produto, exemplo: objetivo do tratamento)
\end{itemize}

O sistema deverá manter o resgistro de localidades específicas como por exemplo:
bairros, para que seja possível realizar consultas de produtos e representantes
a partir de tais localidades utilizando um mapa interativo, exemplos:
``em qual bairro há determinado produto?'' ou ``quantos revendedores estão em tal bairro?'',
etc.

O sistema deverá ser capaz de apresentar um conjunto de revendedores a
partir de localidade, distância, etc. Em cada consulta realizada o sistema
deverá apresentar a distância em relação ao objeto procurado (produto ou
representante, por exemplo).

Localidades que possuam duas dimensões serão
representadas como polígonos, de acordo com o modelo de dados espaciais enquanto
os Representantes serão representados por pontos.

O sistema deverá ser capaz de realizar levantamento de quais regiões possuem maior
quantidade de revendedores e produtos.

O cliente poderá se comunicar com um Representante através de um chat após realizar
uma busca.

Os Representantes também poderão registrar vendas realizadas caso a concretizem
especificando os produtos e a localização da venda.
\clearpage
\section{Requisitos}
\subsection{Requisitos Funcionais}
A seguir segue a tabela com o conjunto de requisitos funcionais:
\begin{center}
    \begin{tabular}{| l | p{8cm} |}
        \hline
        CÓD. REQUISITO & DESCRIÇÃO \\ \hline
        RF001 & Registro de Representantes \\ \hline
        RF002 & Registro de Produtos \\ \hline
        RF003 & [ESP]Registro das superfícies de Localiades relevantes no BD geográfico \\ \hline
        RF004 & [ESP]O sistema deve fornecer um mapa interativo para consultas espaciais \\ \hline
        RF005 & [ESP]Consulta de Produtos por Representantes através do mapa interativo \\ \hline
        RF006 & [ESP]Consulta de Representantes através do mapa interativo  \\ \hline
        RF007 & [ESP]Consulta por Localidade \\ \hline
        RF008 & Consulta de Produtos da empresa \\ \hline
        RF009 & Levantamento de pesquisas realizadas (por produto e localidade)\\
        \hline
    \end{tabular}
\end{center}

\end{document}
